\chapter{Normativa española de seguridad aérea}
\label{chap:normativa}

\drop{L}{}a normativa temporal española para regular la utilización civil de las \acs{RPA}s se contiene en los artículos 50 y 51 de la Ley 18/2014, de 15 de octubre, de aprobación de medidas urgentes para el crecimiento, la competitividad y la eficiencia, publicada en el boletín oficial del estado nº 252, del 17 de octubre de 2014. Esta normativa ha sido extraída íntegramente del documento «\textit{Los Drones y sus aplicaciones a la ingeniería civil}» \cite{normativa}.  

\section{Modificaciones a la Ley de Navegación Aérea}

En el artículo 51 se introducen modificaciones a la Ley 48/1960, de 21 de julio, sobre navegación Aérea (\acs{LNA}), para adaptarla a las aeronaves pilotadas por control remoto. En particular:

\begin{itemize}
\item Se modifica la definición de aeronave del artículo 11 de la \acs{LNA} para establecer sin lugar a dudas que las aeronaves pilotadas por control remoto o drones son aeronaves.
\item Se modifica el artículo 150 de la \acs{LNA} para especificar que las aeronaves civiles pilotadas por control remoto, cualesquiera que sean las finalidades a las que se destinen excepto las que sean utilizadas exclusivamente con fines recreativos o deportivos, quedarán sujetas asimismo a lo establecido en la \acs{LNA} y en sus normas de desarrollo, y que no están obligadas a utilizar infraestructuras aeroportuarias autorizadas.
\item Se modifica el artículo 151 de la \acs{LNA} para permitir que las actividades de trabajos técnicos y científicos realizadas
con aeronaves pilotadas por control remoto puedan iniciarse con una comunicación previa a la Agencia Estatal de Seguridad Aérea (AESA) conforme al artículo 71 bis de la Ley 30/1992 de 26 de noviembre de régimen jurídico de las Administraciones Públicas y del Procedimiento Administrativo Común, sin necesidad de autorización explícita.
\end{itemize}

\section{Normas para la operación de aeronaves civiles pilotadas por control remoto}

El artículo 50 contiene las disposiciones que regulan la utilización civil de las aeronaves pilotadas por control remoto. En su primer apartado se deja sentada la responsabilidad del operador sobre la aeronave y su operación, y se hace referencia a la obligación de éste de cumplir con todo el resto de normas que sean aplicables, en particular las que se refieren a la utilización del espectro radioeléctrico, y, en su caso, la protección de datos y la toma de imágenes aéreas, de las que se habla en el apartado 5 posterior.

En el apartado 2 del artículo 50 se establece la exención para las aeronaves civiles pilotadas por control remoto cuya masa máxima al despegue no exceda de 25 kg de los requisitos establecidos con carácter general para todas las aeronaves en los artículos 29 y 36 de la \acs{LNA} de inscribirse en el registro de Matrícula de Aeronaves y disponer de certificado de aeronavegabilidad, conforme a lo ya previsto en el artículo 151 de esta misma Ley.

Por otro lado, en el mismo apartado se establece la obligación para todas las aeronaves pilotadas por control remoto de llevar fijada a su estructura una placa de identificación en la que figure, de forma legible a simple vista e indeleble, la identificación de la aeronave, mediante la designación específica y, en su caso, número de serie, así como el nombre de la empresa operadora y sus datos de contacto.

Las actividades aéreas con aeronaves pilotadas por control remoto que se contemplan en este artículo 50 son:

\begin{itemize}
\item Trabajos técnicos o científicos, ya sea por cuenta ajena o por cuenta propia; los requisitos para su realización y los escenarios operacionales en que se pueden realizar están contemplados en el apartado 3 del artículo 50.
\item Vuelos especiales, cuyos requisitos se contemplan en el apartado 4 del artículo 50.
\end{itemize}

En todos los casos, trabajos aéreos y vuelos especiales, se establece la limitación de que los vuelos habrán de realizarse de día, en condiciones meteorológicas visuales y en espacio aéreo no controlado.

\section{Normas para las operaciones de trabajos aéreos}

Las operaciones de trabajos técnicos o científicos están sujetas además a las siguientes condiciones y limitaciones:
\begin{itemize}
\item Las aeronaves pilotadas por control remoto cuya masa máxima al despegue no exceda de 25 kg sólo podrán operar en zonas fuera de aglomeraciones de edificios y personas en ciudades, pueblos o lugares habitados, y a una altura sobre el terreno no mayor de 120 metros. Estas aeronaves habrán de volar dentro del alcance visual del piloto, a una distancia de éste no mayor de 500 metros. Se exceptúan aquéllas cuya masa máxima al despegue sea inferior a 2 kg, que podrán operar más allá del alcance visual del piloto, siempre que se mantengan dentro del alcance de la emisión por radio de la estación de control y que cuenten con medios para poder conocer la posición de la aeronave.
\item El resto de las aeronaves, podrán operar con las condiciones establecidas en su certificado de aeronavegabilidad emitido por la Agencia Estatal de Seguridad Aérea.
\end{itemize}

Los operadores de trabajos técnicos o científicos con aeronaves pilotadas por control remoto habrán de cumplir los siguientes 10 requisitos:
\begin{itemize}
\item Disponer de la documentación relativa a la caracterización de las aeronaves que vaya a utilizar.
\item Haber elaborado un «Manual de Operaciones del operador» que establezca los procedimientos de la operación. Este documento no debe confundirse con el «Manual de Vuelo» o documento equivalente de la aeronave, que explica su funcionamiento y da instrucciones para su manejo, incluyendo las situaciones anormales y de emergencia, sino que debe contener los criterios y procedimientos que va a utilizar el operador para realizar de manera segura los diferentes tipos de operaciones que lleve a cabo.
\item Haber realizado un estudio aeronáutico de seguridad de la operación u operaciones, en el que se constate que la misma puede realizarse con seguridad, que puede ser específico para un área geográfica o tipo de operación determinado, o genérico de manera que abarque un abanico amplio de tipos de operación y/o áreas geográficas.
\item Haber realizado con resultado satisfactorio los vuelos de prueba necesarios para demostrar que la operación pretendida puede realizarse con seguridad.
\item Haber establecido un programa de mantenimiento de la aeronave, de acuerdo a las recomendaciones del fabricante.
\item Que la aeronave esté pilotada por pilotos que cumplan los requisitos establecidos en el apartado 5 del mismo artículo 50.
\item Que el operador cuente con una póliza de seguro u otra garantía financiera que cubra la responsabilidad civil frente a terceros por daños que puedan surgir durante y por causa de la ejecución del vuelo, conforme a la normativa aplicable.
\item Haber adoptado las medidas adecuadas para proteger a la aeronave de actos de interferencia ilícita durante las operaciones y establecido los procedimientos necesarios para evitar el acceso de personal no autorizado a la estación de control y al lugar de almacenamiento de la aeronave.
\item Haber adoptado las medidas adicionales necesarias para garantizar la seguridad de la operación y para la protección de las personas y bienes subyacentes.
\item No volar en ningún caso a menos de 8 km de cualquier aeropuerto o aeródromo, o si se trata de una operación más allá del alcance visual del piloto de una aeronave de menos de 2 kg y el aeropuerto cuenta con procedimientos de vuelo instrumental, a menos de 15 km de su punto de referencia.
\end{itemize}
\clearpage
\section{Normas para vuelos especiales}

Los vuelos especiales que se contemplan son:

\begin{itemize}
\item Vuelos de prueba de producción o de mantenimiento, realizados por los fabricantes u organizaciones dedicadas al mantenimiento.
\item Vuelos de demostración no abiertos al público, dirigidos a grupos cerrados de asistentes a un evento o clientes potenciales de un fabricante u operador.
\item Vuelos para programas de investigación, en que se intenta demostrar la viabilidad de realizar determinadas operaciones con aeronaves de este tipo.
\item Vuelos de desarrollo, para poner a punto técnicas y procedimientos para poner en producción una actividad con este tipo de aeronaves.
\item Vuelos de I+D realizados por fabricantes de aeronaves pilotadas por control remoto para el desarrollo de nuevos productos.
\item Los necesarios para demostrar que las actividades de trabajos aéreos solicitadas conforme al apartado 3 pueden realizarse con seguridad.
\end{itemize}

Además de la limitación general mencionada anteriormente, estos vuelos habrán de realizarse en zonas fuera de aglomeraciones de edificios en ciudades, pueblos o lugares habitados o de reuniones de personas al aire libre, y dentro del alcance visual del piloto, o, en caso contrario, en una zona del espacio aéreo segregada al efecto.

Los requisitos para su realización son los mismos que para las operaciones de trabajos aéreos, enumerados en la sección anterior, con excepción de los puntos 2, 4 y 5. Por el contrario, han de cumplir con un requisito adicional, que consiste en establecer una zona de seguridad en relación con la zona de realización del vuelo.

\section{Situaciones de riesgo, catástrofe o calamidad pública}

Los operadores de trabajos aéreos habilitados conforme a la normativa para realizar esas actividades podrán realizar, bajo su responsabilidad, vuelos que no se ajusten a las condiciones y limitaciones mencionadas anteriormente para las operaciones de trabajos aéreos y para los vuelos especiales, en situaciones de grave riesgo, catástrofe o calamidad pública, así como para la protección y socorro de personas y bienes en los casos en que dichas situaciones se produzcan, cuando les sea requerido por las autoridades responsables de la gestión de dichas situaciones.

\clearpage

\section{Requisitos aplicables a los pilotos de las aeronaves civiles pilotadas por control remoto}

Se contienen en el apartado 5 del artículo 50, y consisten en:

\begin{itemize}
\item Acreditar que poseen los conocimientos teóricos correspondientes a cualquier licencia de piloto, lo que puede hacerse de una de estas tres maneras:
	\begin{itemize}
	\item Siendo titular de una licencia, o habiéndolo sido dentro de los últimos 5 años.
	\item Demostrando de forma fehaciente que disponen de los conocimientos teóricos necesarios para la obtención.
	\item Para aeronaves de hasta 25 kg de masa máxima al despegue, disponer de un certificado básico para el pilotaje de aeronaves pilotadas por control remoto, expedido por una Organización de Formación Aprobada por AESA conforme al Reglamento de Personal de Vuelo de la Comisión Europea.
	\end{itemize}
\item Quienes no sean titulares de una licencia de piloto deberán acreditar además:
	\begin{itemize}
	\item Tener 18 años de edad cumplidos.
	\item Ser titulares de un certificado médico correspondiente al menos al requerido para una licencia de piloto de aviación ligera, para aeronaves de hasta 25 kg, y de Clase 2, para las de masa superior, conforme al citado Reglamento de Personal de Vuelo de la Comisión Europea.
	\end{itemize}
\item Además, todos los pilotos deberán disponer de un documento que acredite que disponen de los conocimientos adecuados de la aeronave que van a pilotar y sus sistemas, así como de su pilotaje.
\end{itemize}


\chapter{Resumen}

\drop{E}{}n este proyecto se analizan los vehículos aéreos no tripulados como parte fundamental para la ayuda, a los equipos de emergencia, en labores relacionadas con situaciones de catástrofe, bien sea natural o artificial. Las situaciones de desastre pueden ser el escenario ideal para trabajar con drones, debido a las ventajas que proporcionan en este tipo de entornos.

Por esta razón, se desarrolla un sistema adaptativo que permita la coordinación de distintos \acs{UAV}s en entornos que han sido afectados por una catástrofe. El sistema es capaz de monitorizar diferentes puntos de localización, según una prioridad determinada por los equipos de rescate, desplegar los drones que los equipos de emergencia consideren oportunos, acudir a los puntos de interés de forma coordinada y recolectar imágenes sobre el terreno que posteriormente serán analizadas de manera automática. Para llevar a cabo las pruebas del sistema se ha contado con el vehículo aéreo \textit{3DR IRIS+}, que puede ser programado gracias al entorno de programación desarrollado por 3D Robotics, \textit{DroneKit-Python}. Gracias a estos dos elementos el sistema es capaz de hacer volar el dron a cualquier punto de interés, pero para la captura de imágenes es necesario el uso de la librería \textit{OpenCV}. Por último, las imágenes que han sido capturadas, en las diferentes zonas que los equipos de rescate han considerado oportunas, son analizadas por \textit{Google Cloud Vision \acs{API}}, obteniendo información complementaria del estado del territorio castigado por la catástrofe.

Mediante el uso de este sistema se espera conseguir una mejor productividad y eficiencia en las labores de rescate. Los drones pueden realizar un análisis aéreo del entorno en mucho menos tiempo que el personal de rescate terrestre. Además, la visión aérea ayuda a tomar decisiones más adecuadas y evita la exposición al peligro por parte del personal de emergencia.

\chapter{Abstract}

In this project, \acs{UAV}s are analysed as a key part to aid emergency teams in either natural or artificial disaster-related situations. The forementioned situations may be the ideal environment to work with drones because of the advantages they provide in such environments.


For that reason, an adaptative system that allows coordination of different \acs{UAV}s in disaster affected environments will be developed. This system is able to monitor different location points depending on a priority previously specified by rescue teams, deploy drones that emergency teams have deemed fit, go to the points of interest in a coordinated manner, and collect images on the ground which will then be analysed automatically. To carry out system testing has been used aerial vehicle \textit{3DR IRIS+}, which can be programmed through the programming environment developed by 3D Robotics, \textit{DroneKit-Python}. Thanks to these two elements the system is able to fly the drone to any point of interes, whereas the use of the \textit{OpenCV} Library is necessary when it comes to video capturing. Finally, the images that have been captured in the different areas considered appropriate by by the rescue teams, will be analysed by \textit{Google Cloud Vision \acs{API}}, obtaining additional information about the state of the disaster-damaged territory.

By using the system, a productivity and efficiency improvement in rescue efforts is expected to be achieved. Drones can perform an aerial analysis of the environment in much less time than rescue personnel on the ground. In addition, the aerial view is helpful to make better decisions, and avoids the emergency personnel's exposure to danger.

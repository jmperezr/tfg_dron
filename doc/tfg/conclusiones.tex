\chapter{Conclusiones y trabajo futuro}
\label{chap:conclusiones}

\drop{E}{}n este capítulo se presentan las conclusiones que se pueden extraer después de todo el trabajo que se ha realizado en el proyecto. Se justificarán las competencias abordadas en el \acs{TFG}. De igual modo, se propondrá un conjunto de tareas que pueden servir como posibles líneas de trabajo futuro.

\section{Conclusiones}
\label{sec:conclusiones}

La \textbf{consecuencia principal} de este proyecto es el desarrollo de un sistema capaz de monitorizar ciertos puntos de interés, mediante la coordinación de varios \acs{UAV}s, en vuelo, que retransmiten imágenes en directo de la zona afectada por una catástrofe, de modo que, los servicios de emergencia reciban una ayuda adicional que permita mejorar los tiempos y la eficiencia de las misiones en esta clase de situaciones. A continuación, las conclusiones serán alineadas con la consecución de los objetivos planteados en el Capítulo~\ref{chap:objetivos}.

\subsection{Objetivo general cumplido}

Para conseguir este fin, se ha hecho uso del vehículo aéreo \textbf{3DR IRIS+}. Tras finalizar el proyecto, su compra se puede considerar un \textbf{gran acierto}, debido a que, su guiado automático es intuitivo y sencillo en su configuración. Además, implementa medidas de seguridad que son fundamentales para pilotos que, como en el supuesto del \acs{TFG}, no disponen de experiencia. Permitiendo así, que mediante el accionamiento de una palanca, el dron vuelva al punto en el que realizó el despegue o realice automáticamente un aterrizaje. 

La utilización de \textbf{DroneKit-Python}, como plataforma de desarrollo de software, es otra de las \textbf{ventajas} y de los motivos por los que comprar el 3DR IRIS+ ha sido todo un acierto. DroneKit-Python cuenta con un amplio abanico de ejemplos y funciones para realizar, de una manera simple, cualquier tipo de operación con el dron. Tanto es así, que permite llevar a cabo procedimientos como la conexión con el vehículo, el despegue, el aterrizaje, la elaboración de misiones, etc.

Al final del proyecto se consigue desarrollar un \textbf{sistema adaptativo} capaz de desplegar y coordinar \acs{UAV}s que recogen información en un entorno afectado por algún tipo de catástrofe. 

El sistema logra:

\begin{itemize}
\item \textbf{Reducir la exposición de agentes humanos}: con la ayuda de drones se posibilita inspeccionar la zona sin poner en peligro a ningún integrante del equipo de emergencia. Además, la visión aérea proporcionada por el vehículo contribuye a la realización de mejores planes de rescate, reduciendo así la exposición al peligro. 
\item \textbf{Disminuir el tiempo} empleado en recoger información del lugar afectado por la catástrofe: en los casos de estudio llevados a cabo en el proyecto, se comprueba como la utilización de varios \acs{UAV}s puede reducir el tiempo, para la obtención de imágenes, hasta en un tercio del tiempo que se gasta con un solo dron.
\item \textbf{Minimizar el daño} resultante global: con todo lo anterior, se posibilita una mejor y más rapida toma de decisiones, de tal manera que, el daño global es mucho menor utilizando el sistema.
\end{itemize}

Cabe destacar la \textbf{dificultad} que entraña al proyecto el empleo de varios \textbf{dispositivos hardware}. Al desarrollo del código se le une la complejidad, y el esfuerzo en tiempo, de estudiar y comprender el funcionamiento de los componentes. Más aún, si entre estos dispositivos se encuentra un dron, dado que, en las diferentes pruebas de vuelo que se deben realizar es más que probable que el aparato sufra alguna caída o golpe que pueda causar un determinado deterioro en alguno de sus componentes. Estos imprevistos, relacionados con la avería del hardware, provocan que el desarrollo se alargue, a consecuencia de que, se debe esperar hasta el envío de repuestos nuevos.

\subsection{Objetivos específicos cumplidos}

\begin{itemize}
\item Se establece un modo de \textbf{precisar puntos de monitorización} mediante archivos XML en los que se indica la ubicación del punto de interés, la altura que debe alcanzar el \acs{UAV} en ese punto y la prioridad de volar hasta él.
\item El sistema es totalmente \textbf{escalable} en cuanto al número de vehículos a utilizar. En el caso de hacer uso de un solo dron el coordinador se abstrae de esta situación y trabaja del mismo modo que si dispusiera de \textit{N} drones.
\item Se \textbf{plantea la coordinación}, con el fin de que, ningún vehículo se encuentre en estado ocioso mientras el \textit{Pool de Misiones} no esté vacío.
\item El sistema \textbf{proporciona capacidad de análisis forense} por medio de la utilización de un sistema \acs{FPV}, compuesto entre otros por una cámara GoPro HERO 3+.
\item Se implementa el \textbf{análisis de imágenes} a través de Google Cloud Vision \acs{API}. Es interesante contar con una plataforma que posibilita el análisis de las imágenes que la cámara, incorporada en el dron, captura. Cualquier clase de información complementaria, aunque parezca insignificante, puede ser una gran contribución a los equipos de emergencia.
\end{itemize}

\section{Trabajos futuros}
\label{sec:trabajos futuros}

Con el fin de proseguir el desarrollo del proyecto, partiendo de la base de este \acs{TFG}, se proponen varias líneas de trabajo:

\begin{itemize}
\item \textbf{Mejora del algoritmo de coordinación utilizando un enfoque descentralizado}: descentralizar la manera en la que se realiza la coordinación añadiría robustez al sistema, en cuanto a que, dispondría de capacidad para no depender de una única máquina. Para ello, puede ser adecuado el uso de un servidor en \textbf{ZeroC \acs{ICE}} como plataforma de comunicación.
\item \textbf{Implementar detección de movimiento}: una funcionalidad más que interesante sería que el sistema fuera capaz de detectar cualquier tipo de movimiento existente en las imágenes. La identificación de supervivientes en situaciones de emergencia resultaría mucho más sencilla. Sin embargo, este tipo de implementación es costosa, a causa de que, el dron permanece en constante movimiento y por ello no se posee una imagen fija del entorno. Por lo tanto, sería necesario entrenar al sistema mediante modelos de aprendizaje para así poder detectar el movimiento.
\item \textbf{Aumentar la flota de vehículos}: sería importante ampliar el número de drones del que se dispone, de manera que, se pueda plantear la coordinación de los \acs{UAV}s en entornos reales y no solo simulados.
\end{itemize}



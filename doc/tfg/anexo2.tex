\chapter{DroneKit-Python: guía y buenas prácticas}
\label{chap:dronekitguia}

\drop{E}{}sta guía ofrece una visión general de cómo utilizar DroneKit-Python mediante un conjunto de buenas prácticas recomendadas. DroneKit-Python se comunica con los pilotos automáticos de los vehículos mediante el protocolo MAVLink, que define cómo se envían los comandos y la configuración de telemetría entre los vehículos, estaciones terrestres y sistemas en otros una red MAVLink.

\section{Conexión}

En la mayoría de los casos se debe usar la forma normal para conectar a un vehículo, es decir, se debe establecer el parámetro \textit{wait\_ready = True} para asegurar que el vehículo cuente con todos los atributos una vez que la conexión se ha realizado.

\begin{listing}[
 float=h!,
 language = Python,
 caption = {Conexión con el vehículo mediante DroneKit-Python},
 label  = code:conexion]
from dronekit import connect

# Connect to the Vehicle (in this case a UDP endpoint)
vehicle = connect('REPLACE_connection_string_for_your_vehicle', wait_ready=True)
\end{listing}

La llamada a la conexión a veces puede fallar con una excepción. Información adicional acerca de la excepción puede ser obtenida mediante la ejecución de la llamada a la conexión dentro de un bloque try-catch (ver Listado~\ref{code:conexiontry}).

Si una conexión se realiza correctamente desde una estación terrestre, pero no desde DroneKit-Python puede ser que su velocidad de transmisión sea incorrecta para su hardware. Esta tasa también se puede configurar en el método \textit{connect()}.

\clearpage

\begin{listing}[
 float=h!,
 language = Python,
 caption = {Conexión haciendo uso de try-catch},
 label  = code:conexiontry]
import dronekit
import socket
import exceptions

try:
    dronekit.connect('REPLACE_connection_string_for_your_vehicle', heartbeat_timeout=15)

# Bad TCP connection
except socket.error:
    print 'No server exists!'

# Bad TTY connection
except exceptions.OSError as e:
    print 'No serial exists!'

# API Error
except dronekit.APIException:
    print 'Timeout!'

# Other error
except:
    print 'Some other error!'
\end{listing}

 
\section{Secuencia de inicio}

Generalmente, se debe utilizar la secuencia de lanzamiento estándar que consiste en:

\begin{itemize}
\item Consultar el parámetro \textit{Vehicle.is\_armable} hasta que el vehículo está listo para ser armado.
\item Ajustar el \textit{Vehicle.mode} como \textit{GUIDED}.
\item Establecer \textit{Vehicle.armed} a \textit{True} y consultar el mismo atributo hasta que el vehículo se encuentre armado.
\item Llamar a \textit{Vehicle.simple\_takeoff} con la altitud deseada.
\item Consultar la altitud y solo permitir que el código continúe sólo cuando se alcanza dicha altura.
\end{itemize}

Este planteamiento garantiza que los comandos se envían sólo al vehículo cuando es capaz de actuar sobre ellos.
\clearpage

\begin{listing}[
 float=h!,
 language = Python,
 caption = {Secuencia de inicio},
 label  = code:despegue]
# Connect to the Vehicle (in this case a simulator running the same computer)
vehicle = connect('tcp:127.0.0.1:5760', wait_ready=True)

def arm_and_takeoff(aTargetAltitude):
    """
    Arms vehicle and fly to aTargetAltitude.
    """

    print "Basic pre-arm checks"
    # Don't try to arm until autopilot is ready
    while not vehicle.is_armable:
        print " Waiting for vehicle to initialise..."
        time.sleep(1)

    print "Arming motors"
    # Copter should arm in GUIDED mode
    vehicle.mode    = VehicleMode("GUIDED")
    vehicle.armed   = True

    # Confirm vehicle armed before attempting to take off
    while not vehicle.armed:
        print " Waiting for arming..."
        time.sleep(1)

    print "Taking off!"
    vehicle.simple_takeoff(aTargetAltitude) # Take off to target altitude

    # Wait until the vehicle reaches a safe height before processing the goto (otherwise the command
    #  after Vehicle.simple_takeoff will execute immediately).
    while True:
        print " Altitude: ", vehicle.location.global_relative_frame.alt
        #Break and return from function just below target altitude.
        if vehicle.location.global_relative_frame.alt>=aTargetAltitude*0.95:
            print "Reached target altitude"
            break
        time.sleep(1)

arm_and_takeoff(20)
\end{listing}

\section{Misiones y «waypoints»}

DroneKit-Python puede crear y modificar misiones autónomas. Si bien es posible construir aplicaciones con DroneKit-Python de forma dinámica, mediante la construcción de misiones «on the fly», 3D Robotics recomienda utilizar el modo guiado para aplicaciones con ArduCopter.

El modo \textit{AUTO} se utiliza para la ejecución de misiones con «waypoints» predefinidos.

DroneKit-Python proporciona funciones básicas para descargar y borrar los comandos actuales de la misión del vehículo, agregar y cargar nuevos comandos de misión y contar el número de «waypoints» visitados. Es una buena idea apoyarse en estas primitivas básicas para crear la funcionalidad de planificación de la misión de alto nivel.

\begin{listing}[
 float=h!,
 language = Python,
 caption = {Creación de misiones},
 label  = code:misiones]
# Connect to the Vehicle (in this case a simulated vehicle at 127.0.0.1:14550)
vehicle = connect('127.0.0.1:14550', wait_ready=True)

# Get the set of commands from the vehicle
cmds = vehicle.commands
cmds.download()
cmds.wait_ready()

# Clear the current mission (command is sent when we call upload())
cmds.clear()

# Create and add commands
cmd1=Command( 0, 0, 0, mavutil.mavlink.MAV_FRAME_GLOBAL_RELATIVE_ALT, mavutil.mavlink.MAV_CMD_NAV_TAKEOFF, 0, 0, 0, 0, 0, 0, 0, 0, 10)
cmd2=Command( 0, 0, 0, mavutil.mavlink.MAV_FRAME_GLOBAL_RELATIVE_ALT, mavutil.mavlink.MAV_CMD_NAV_WAYPOINT, 0, 0, 0, 0, 0, 0, 10, 10, 10)
cmds.add(cmd1)
cmds.add(cmd2)
cmds.upload() # Send commands
\end{listing}

Para iniciar una misión, solo se debe cambiar el modo a \textit{AUTO}.

\begin{listing}[
 float=h!,
 language = Python,
 caption = {Iniciar misión},
 label  = code:iniciomision]
# Connect to the Vehicle (in this case a simulated vehicle at 127.0.0.1:14550)
vehicle = connect('127.0.0.1:14550', wait_ready=True)

# Set the vehicle into auto mode
vehicle.mode = VehicleMode("AUTO")
\end{listing}

\section{Pausar el script cuando no se necesita}

Pausar la secuencia de comandos puede reducir la carga del computador.

Por ejemplo, a bajas velocidades es posible que sólo tenga que comprobar si se ha alcanzado un objetivo cada pocos segundos. Usando \textit{time.sleep(2)} entre comprobaciones será más eficiente que comprobarlo más a menudo.

\section{Finalizar el script}

Los scripts deben llamar a \textit{Vehicle.close()} antes de salir para asegurarse de que todos los mensajes se han vaciado antes de que el script finalice.

\begin{listing}[
 float=h!,
 language = Python,
 caption = {Cerrar la conexión con el vehículo},
 label  = code:cerrarvehiculo]
# About to exit script
vehicle.close()
\end{listing}
